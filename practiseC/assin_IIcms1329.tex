\documentclass[11pt]{article}
\usepackage{parskip}
%\setlength{\parindent}{15pt}
\begin{document}
\begin{center}
{\huge This is title}
\vspace{10 mm}
{\\ \Large{Balchandra pujari}}
\vspace{5 mm}
{\\ \normalsize{January 15,2014}}
\vspace{10 mm}
\small \bf{\\Abstract}
\end{center}
\begin{flushleft}
\hspace{3 mm}
{This is an abstract.There is an {\it environment} to creat a abstract and one to creat this title page.}
\end{flushleft}
\vspace{3 mm}
\begin{flushleft}
\section{\Large\bf This is the first `section'}
\subsection{\normalsize This is first `sub-section'}
\subsubsection{\small This is first `sub-sub-section'}
 I have created this article as an example. The goal is to learn the \LaTeX{}. \break Now note that I am going to change to the next paragraph.
\par \hspace{3mm} This is the paragraph and not just a new line. Note that there is so \break called `indentation'. Also note that when you give the quotation marks \nolinebreak they \newline appear correct.Below is an equation(with the number)
\end{flushleft}
\begin{center}
\begin{equation}
\frac{\partial \Psi}{\partial \vec{x}}=\sin({\theta})
\end{equation}
\end{center}
\begin{flushleft}
{\normalsize But the following does not have a number (and note that I did not \\ use \textbackslash {\bf nonnumber} command)}
\end{flushleft}
\begin{center}
$\lim \limits_{x \rightarrow \infty} \int{x^2}{e^{\frac{x}{2}}}=0$
\end{center}
\begin{flushleft}
{(too many backets?)}
\end{flushleft}
\begin{flushright}
{Here is the text that is aligned to right!\\This is the new line on the right side.}
\end{flushright}
\end{document}


